\chapter{Creating single shot detector}
\label{chapt:improvemnets}

\section{base network}
resnet
bla bla 10,18,32,50,101...

feature map sized
block1: 56x56
block2: 28x28
block3: 14x14
block4: 7x7

unlike original SSD on VGG-16 we decided to start with this basic setup without adding more feature layers, because we are interested in small objects on surveillance video and not for recognizing large objects on photographs.

\begin{figure}
    \centering
    \includegraphics[width=\textwidth]{img/resnet_blocks.png}
    \caption{High level structure of resnet with SSD}
    \label{fig:my_label}
\end{figure}




\section{artificial dataset}
random generated circles, triangles and squares with random size and color, triangles are randomly rotated. placing random object over a background of random lines generated by using the same types of objects but upscaled so they are not fully in the image. Object for detection is drawn on top. 

\section{Simple detector for one object}

optimizer = Adam(lr=0.001, betas=(0.9, 0.999), eps=1e-08, weight decay=0)
class loss = CrossEntropyLoss
loc loss = MSELoss output of NN with absolute coordinates of ground truth box normalized to -1,1 

resnet 18: in addition to fully connected layer we use feature map before avg pooling as input to 7x7 convolution with 4 filters (4 coordinates)

results: class acc: over 99\%
class loss: 0.02
location loss (mean squared error): 0.005


\begin{figure}
    \centering
    \includegraphics[width=\textwidth]{img/simple_detection.png}
    \caption{Detections by simple model 1 7x7 feature map to 4 coordinates and fully connected for class}
    \label{fig:my_label}
\end{figure}

