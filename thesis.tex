%%% he main file. It contains definitions of basic parameters and includes all other parts.

%% Settings for single-side (simplex) printing
% Margins: left 40mm, right 25mm, top and bottom 25mm
% (but beware, LaTeX adds 1in implicitly)
%\documentclass[12pt,a4paper]{report}
%\setlength\textwidth{145mm}
%\setlength\textheight{247mm}
%\setlength\oddsidemargin{15mm}
%\setlength\evensidemargin{15mm}
%\setlength\topmargin{0mm}
%\setlength\headsep{0mm}
%\setlength\headheight{0mm}
% \openright makes the following text appear on a right-hand page
%\let\openright=\clearpage

%% Settings for two-sided (duplex) printing
% urcite tiskni oboustranne, jednostranny bakalarky jsou fuj.
\documentclass[12pt,a4paper,twoside,openright]{report}
%
% btw haha, tady si ruzny matfyzaci mysleli ze umej typografii lip nez
% profesionalove co se tim zabejvaji poslednich 300 let. Kdyby ti prislo ze
% stranky jsou tedka "divne rozjety do stran", tak to je dobre a kazdej tiskar
% to pochvali. Pak nekdy kdyztak vysvetlim proc to tak je.
%
%\setlength\textwidth{145mm}
%\setlength\textheight{247mm}
%\setlength\oddsidemargin{14.2mm}
%\setlength\evensidemargin{0mm}
%\setlength\topmargin{0mm}
%\setlength\headsep{0mm}
%\setlength\headheight{0mm}
\let\openright=\cleardoublepage

%% Prefer Latin Modern fonts
\usepackage{lmodern}

%% Further useful packages (included in most LaTeX distributions)
\usepackage{amsmath}        % extensions for typesetting of math
\usepackage{amsfonts}       % math fonts
\usepackage{amsthm}         % theorems, definitions, etc.
\usepackage{bbding}         % various symbols (squares, asterisks, scissors, ...)
\usepackage{bm}             % boldface symbols (\bm)
\usepackage{bbm}
\usepackage{graphicx}       % embedding of pictures
\usepackage{fancyvrb}       % improved verbatim environment
\usepackage[numbers]{natbib} % citation style AUTHOR (YEAR), or AUTHOR [NUMBER]
\bibliographystyle{plainnat} % this must be moved here for natbib to work correctly
\usepackage[nottoc]{tocbibind} % makes sure that bibliography and the lists
			    % of figures/tables are included in the table
			    % of contents
\usepackage{dcolumn}        % improved alignment of table columns
\usepackage{booktabs}       % improved horizontal lines in tables
\usepackage{paralist}       % improved enumerate and itemize
\usepackage{multicol}
\usepackage{multirow, makecell}
% \usepackage[usenames]{xcolor}  % typesetting in color
\usepackage[table,dvipsnames,usenames,rgb]{xcolor}  % typesetting in color

\usepackage[hyphens,spaces,obeyspaces]{url}

\usepackage[textsize=tiny]{todonotes} %visual todo!

\usepackage{xspace}         %for......xspace.
\usepackage{fancyvrb}
\usepackage{rotating}
\usepackage{collcell}
\usepackage{nomencl}
\usepackage{siunitx}
\usepackage{listings}

% my packages
\usepackage{models}

\usepackage{pgfplots}
\pgfplotsset{compat=1.8}
\usepgfplotslibrary{statistics}


%% Generate PDF/A-2u
\usepackage[a-2u]{pdfx}

\usepackage{cleveref}       %\cref yay!

%% Character encoding: usually latin2, cp1250 or utf8:
\usepackage[utf8]{inputenc}

%%% Basic information on the thesis

% Thesis title in English (exactly as in the formal assignment)
\def\ThesisTitle{Object detection for video surveillance using the SSD approach}

% Author of the thesis
\def\ThesisAuthor{Marek Dobranský}

% Year when the thesis is submitted
\def\YearSubmitted{2019}

% Name of the department or institute, where the work was officially assigned
% (according to the Organizational Structure of MFF UK in English,
% or a full name of a department outside MFF)
\def\Department{Department of Software Engineering}

% Is it a department (katedra), or an institute (ústav)?
\def\DeptType{Department}

% Thesis supervisor: name, surname and titles
\def\Supervisor{RNDr. Jakub Lokoč, Ph.D.}

% Supervisor's department (again according to Organizational structure of MFF)
\def\SupervisorsDepartment{Department of Software Engineering}

% Study program and specialization
\def\StudyProgramme{Computer Science}
\def\StudyBranch{Artificial inteligence}

% An optional dedication: you can thank whomever you wish (your supervisor,
% consultant, a person who lent the software, etc.)
\def\Dedication{%
I would like to express my gratitude to RNDr. Jakub Lokoč, Ph.D., the supervisor of this thesis, for his patient guidance, feedback and all advice he has given me. 

I also want to thank my family and girlfriend for their continued support and encouragement during my studies and especially during the time spent working on this thesis.
}

% Abstract (recommended length around 80-200 words; this is not a copy of your thesis assignment!)
\def\Abstract{%
The surveillance cameras serve various purposes ranging from security to traffic monitoring and marketing. However, with the increasing quantity of utilized cameras, manual video monitoring has become too laborious. In recent years, a lot of development in artificial intelligence has been focused on processing the video data automatically and then outputting the desired notifications and statistics. This thesis studies the state-of-the-art deep learning models for object detection in a surveillance video and takes an in-depth look at SSD architecture. We aim to enhance the performance of SSD by updating its underlying feature extraction network. We propose to replace the initially used VGG model by a selection of modern ResNet, Xception and NASNet classification networks. The experiments show that the ResNet50 model offers the best trade-off between speed and precision, while significantly outperforming VGG. With a series of modifications, we improved the Xception model to match the ResNet performance. On top of the architecture-based improvements, we analyze the relationship between SSD and a number of detected classes and their selection. We also designed and implemented a new detector with the use of temporal context provided by the video frames. This detector delivers enhanced precision while meeting real-time requirements.
}

\def\AbstractSK{
Kamerové systémy dnes slouží různým účelům, od bezpečnosti k monitorování dopravy a marketingu. Nicméně, s rostoucím množstvím kamer se stává manuální sledování videa příliš pracné. V posledních letech se hodně vývoje umělé inteligence zaměřilo na automatické zpracování videa a následný výstup požadovaných upozornění a statistik. Tato práce zkoumá nejmodernější modely hlubokého učení pro detekci objektů v bezpečnostním videu a podrobně se zabývá SSD architekturou. Našim hlavním cílem je zvýšit výkon SSD architektury aktualizací vnitřní sítě extrahující tzv. feature mapy. V práci jsou navrženy vhodné možnosti nahrazení původního VGG modelu pomocí nejnovějších klasifikačních sítí ResNet, Xception a NASNet. Experimentálně jsme zjistili, že model ResNet50 nabízí nejlepší kompromis mezi rychlostí a přesností. Tento model zároveň výrazně překonává VGG. Po zavedení řady modifikací do sítě Xception se nám povedlo dorovnat výkon ResNetu. Kromě vylepšení architektury také analyzujeme vztah mezi SSD a množstvím detekovaných tříd a jejich výběrem. Také jsme navrhli a implementovali nový detektor, který využívá temporální kontext snímku pro detekci objektů. Tento detektor pracuje v reálném čase a se zvýšenou přesností.
}

% 3 to 5 keywords (recommended), each enclosed in curly braces
\def\Keywords{%
{object detection}, {video surveillance}, {deep neural networks}, {SSD architecture}
}

%% The hyperref package for clickable links in PDF and also for storing
%% metadata to PDF (including the table of contents).
%% Most settings are pre-set by the pdfx package.
\hypersetup{unicode}
\hypersetup{breaklinks=true}

% Definitions of macros (see description inside)
\include{utility/macros}

% Title page and various mandatory informational pages
\begin{document}
\include{utility/title}

%%% A page with automatically generated table of contents of the bachelor thesis
\tableofcontents
%%% Each chapter is kept in a separate file
\chapter*{Introduction}
\addcontentsline{toc}{chapter}{Introduction}

In recent years, security cameras have become the most widely used surveillance measure. Other than for security, surveillance cameras are finding their use for multiple purposes such as traffic monitoring and marketing. Thanks to the wide range of applications, surveillance covers most of the public space in the cities. This amount of video streams has become very laborious to monitor. The obvious solution to this problem has presented itself with development in artificial intelligence and the broader availability of computing power. Instead of constant monitoring by people, artificial intelligence can process the video data and then output the desired notifications and statistics.

The technology in the area of automatization in video surveillance has been rapidly changing. The most elementary approaches like motion detection are based on the principle of frame difference. Frame difference methods either compare successive frames with each other or use a background subtraction. It is a fast and simple method for detection of moving objects and is nowadays commonly embedded directly in the security cameras. However, frame difference is prone to false detections as it detects every movement independently on the object type. 

Object detectors with classification are historically based on feature descriptors. The first such algorithm with competitive results was a face detector by \citeauthor{bib:viola}. A few years later, another breakthrough in detection came from \citeauthor{bib:hog} using the histogram of oriented gradients for human detection. The last major leap in the detection, the deep learning, has become the dominant approach in the current decade. In the ImageNet Large Scale Visual Recognition Challenge deep learning approaches have been consistently winning since 2012 and surpassed human performance in 2015.

Current cutting edge deep learning detectors use convolutional neural network architectures. Such networks produce either set of bounding boxes or a pixel by pixel categorization of the desired objects. These detection data can be then used and processed by other applications to produce desired statistics or notifications.

\subsubsection{Goals}
The goal of this thesis is to review state-of-the-art deep learning models used for object detection and implement a selected model. We aim to improve this model for purposes of surveillance by formulating hypotheses based on reviewed techniques and test them experimentally. We have the ambition to design and implement a real-time video detector with the use of temporal information.

Our focus is limited to the detector component of a more extensive detection pipeline. We are not concerned with the pre- and post-processing operations. Although many interesting tasks are based on object detection, e.g., tracking and re-identification,  they are not in the scope of this thesis.

\subsubsection{Thesis Structure}
\Cref{chap:nns} of this thesis begins by defining the metrics needed for evaluation of the performance of neural network detectors. We continue the definitions by presenting a notation used throughout this thesis. We present an overview of convolutional neural networks used for image classification and region based object detectors. 

\Cref{chap:related} is dedicated to provide an in-depth review of real-time detectors, namely SSD and YOLO, and a review of two methods for video detection with temporal information, Tube-CNN and Temporal SSD. 

We present our contributions in \cref{chap:contrib}. We propose a set of improvements of SSD aimed at real-time detection for video surveillance and test them experimentally. Our contributions include a comparison of SSD implemented on a multitude of base networks (ResNet, Xception, NASNet). Then, we tested the relationship between detector performance and a number of detected classes and created the Surveillance dataset for further testing. We also present an improved version of Xception, modified to suit the needs of SSD detector. Our final contribution is an extension of SSD detector by the addition of three-dimensional convolutions in the temporal dimension. We called our detector a Single Shot Detector with Temporal Convolution (SSDTC).

\Cref{chap:exp} provides details on the methodologies used in this work. It also presents supplementary results acquired during the experiments.
\chapter{Standard approaches}

\section{Frame difference}
\section{Edge detection}
\section{Background subtraction}
\section{Optical Flow}
\section{Histogram of oriented gradients}

some others
\chapter{Neural Network Detectors}
\todo{bla}
NNs vs ML
define (NNs) abbr.

\section{Precision Metrics}
mAP, IoU

\section{Classification networks}
\label{chapt:cnets}
In order to better understand detection networks, we will shortly describe their predecessors, classification networks. Classification network is a convolutional neural network (CNN) \cite[ch.~9]{bib:dlbook} that given an image, returns a confidence score of correspondence to each of the classes. Usually, \textit{soft-max} function is applied to the confidence score to represent a probability distribution. The \textit{ImageNet} image recognition challenge\footnote{\url{http://www.image-net.org/challenges/LSVRC/}} is usually used to benchmark the accuracy of such networks.

Later, in \cref{chapt:models}, we will see how a classification model can be used as a backbone for a detector network.

\subsection*{AlexNet (2012)}
A significant breakthrough in use of CNNs happened in 2012, when the \textit{AlexNet} \cite{bib:alexnet} won the \textit{ImageNet} image recognition challenge. It was the first time a deep CNN performed better than traditional computer vision and machine learning approaches. 

AlexNet has a simple architecture with five convolutional layers and two fully connected layers, followed by a softmax layer. It has created a foundation on which today's state-of-the-art models are built and set a new standard for image recognition.

\subsection*{VGG (2014)}
\label{sec:VGG}
The network architecture, mostly known as VGG \cite{bib:vgg}, pushed the concept of AlexNet even further and has proved the feasibility of deep network with small convolutions. 

Each of the VGG's convolutional filters uses 3$\times$3 kernel and the depth of the filters is increased through the network, reaching 512 filters in the last layers. Three fully connected layers and softmax are applied after the convolutions, see \cref{tab:vggarch}. There are multiple versions of the VGG architecture, depending on the number of convolutional layers, the most popular is the 16 layer version. 

VGG network is considered a general architecture for a classification network due to its linear architecture with decreasing size of the features and increasing number of channels. 

\begin{table}[]
    \centering
    \rotatebox{90}{
        \vggArch
    }
    \caption{Architecture of VGG network, version D. Taken from \cite[table 1]{bib:vgg}}
    \label{tab:vggarch}
\end{table}
    
\subsection*{Inception (v1) (2014)}
\label{sec:inception}
Previous architectures showed that increasing the number of layers and layer size, leads to better precision. Inception v1 \cite{bib:googlenet}, also known as GoogLeNet, aims at increasing precision while improving utilization of computing resources.

In order to avoid the growing cost of stacking more layers, the network introduces the concept of sparsity by using the \textit{inception modules}, \cref{fig:incept_mod}. A sparse structure is approximated by using multiple convolutions with different kernel sizes and concatenating the outputs together. Max-pooling is also performed as an alternative to convolutions and concatenated to the output. Because high dimensional convolutions are costly, a reduction to the channels is introduced by using 1$\times$1 convolution as a preceding layer.

The network is then formed by linearly stacking nine inception modules, preceded by a linear stem network and followed by a fully connected classifier. Two auxiliary classifiers are added to intermediate layers of the network to help propagate gradients and provide regularization during the training.

A set of improvements to the Inception network was introduced in later versions of the network. Most notably a factorization of convolution layers was introduced in Inception v2 and v3 \cite{bib:inception2}.


\begin{figure}
    \includegraphics[width=\textwidth]{img/inception}
    \caption{Inception module, picture from \cite[figure 2]{bib:googlenet}.}
    \label{fig:incept_mod}
\end{figure}

\subsection*{ResNet (2015)}
\label{sec:resnet}

Although it has become possible to train deeper and deeper CNNs, degradation has eventually been observed. Adding more layers to models started to produce higher training error. Theoretically, adding more layers to a smaller model should produce at least equal results, as the smaller model is the subspace of the larger one. A solution to this problem was proposed in the ResNet architecture, by directly introducing identity functions to the network \cite{bib:resnet}.  

A baseline of ResNet is directly inspired by VGG (\cref{sec:VGG}). Most of the convolutional layers have 3$\times$3 filters and follow two simple rules: keep the number of filters the same, unless changing the output size and double the filters if the feature size if halved. A residual connection is then added to each pair of the convolutional layers, this connection is either an identity or a projection done by 1$\times$1 convolution to match the increased number of filters. 

\begin{figure}
    \resnetArch
    \caption{Architecture of the ResNet network and residual blocks. Each of the four \textit{Layers} are created by stacking multiple residual blocks.}
    \label{fig:resnet_arch}
\end{figure}

We can see the high level architecture of this model in \cref{fig:resnet_arch} (left). Each of four \textit{Layers} is composed of multiple linearly stacked residual blocks, exact numbers of blocks can be found in \cite[table 1]{bib:resnet}. A feature map size is preserved inside each of the \textit{Layers} and halved between them. A type of residual block we described previously was a \textit{Basic block} with two convolutional layers and is used for smaller ResNet models (ResNet-18, ResNet-34). Deeper ResNet models (ResNet-50, ResNet-101, ResNet-152) use the \textit{Bottleneck block} with three convolutional layers, where the 1$\times$1 layers are responsible for reducing and then restoring dimensions, leaving the 3$\times$3 layer a with smaller input and output dimensions. Remarkably, the 152-layer ResNet has lower complexity than the 16-layer VGG network.



\subsection*{Xception (2017)}
\label{sec:xception}

\subsection*{NasNet (2017)}
\label{sec:nasnet}

\subsection*{Present day networks}
 comparing \cref{fig:cnncomp}

\begin{figure}
    \includegraphics[width=\textwidth]{img/nasnet}
    \caption{Accuracy versus computational demand across top performing published CNN architectures on ImageNet 2012 ILSVRC challenge prediction task. Computational demand is measured in the number of floating-point multiply-add operations to process a single image. Taken from \cite{bib:nasnet}.}
    \label{fig:cnncomp}
\end{figure}

\todo{add comp table for all models}

\section{Detection networks}
\label{chapt:models}

\todo{bla}

\subsection*{Region-based Convolutional Network (R-CNN)}
\subsubsection{R-CNN}
\subsubsection{Fast R-CNN}
\subsubsection{Faster R-CNN}

\subsection*{Mask Region-based Convolutional Network (Mask R-CNN)}

\subsection*{You Only Look Once (YOLO)}
\subsubsection{YOLO}
\label{sec:yolo}
\subsubsection{YOLO v2 and YOLO 9000}

\subsection*{Single-Shot Detector (SSD)}
\label{sec:ssd}
\subsubsection{SSD}

\subsubsection{FSSD}
\subsubsection{RFB}


\section{Use of Detection networks for video processing}

%Optional section
% \section{Other uses of CNNs}
% \subsection{Noise removal/ Regularization}
% \subsection{Image generation}







\chapter{Real-time video detection}
\label{chap:rltm}

real time -> 25fps
this is related work
% \section{Optimizing Video Object Detection via a Scale-Time Lattice}
% \section{Use of Detection networks for video processing}

\section{Detection networks}


\subsection{YOLO: You Only Look Once (2016)}
\label{sec:yolo}
Building on the success of neural network detectors from \textit{R-CNN} family. \citeauthor{bib:yolo} \cite{bib:yolo} introduced a new approach to object detection. They unify networks for localization and classification into a new single network. This network predicts both bounding box positions and class probabilities in a single evaluation. This approach also simplifies the training process, as \textit{YOLO} can be directly trained end-to-end. 

Thanks to straightforward single pass architecture \textit{YOLO} claims to perform at 45 frames per second on \textit{Titan X GPU}. Although it has to sacrifice some precision compared to region proposal methods, it out-performs other real-time systems of its time \cite{bib:overfeat}.


\subsubsection{Detection}
Prediction in \textit{YOLO} work in a grid-based system. It divides the image into \textit{S\x S} grid with each cell responsible for detecting the object centered in that cell.  Each cell produces predictions for \textit{B} bboxes and one set of class confidence predictions. 

Bbox prediction is composed of four positional parameters and confidence score. Center coordinates relate to the grid cell while width and height are represented relative to the whole image. Confidence score reflects IoU with ground-truth box. Class confidence prediction represents the conditional probability of said class, given the presence of the object in that cell. 

Final confidence for each box is the product of both conditional class probabilities and the individual box confidence predictions. We can see the illustration of this process on \cref{fig:yoloDet}.

\begin{figure}
    \centering
    \includegraphics[width=\textwidth]{img/yoloDet}
    \caption{Detection process of \textit{YOLO}. From \cite[fig. 2]{bib:yolo}.}
    \label{fig:yoloDet} 
\end{figure}


\subsubsection{Architecture} 
\textit{YOLO} is designed as a single network that takes the input image and outputs bbox and class predictions. Design of the network is inspired by \textit{Inception} classification network. Although it does not use inception modules, it relies on the 1\x1 reduction layers to speed up 3\x3 convolutions. I\textit{YOLO} t uses 24 convolutional layers followed by two fully connected layers. Full architecture is shown on \cref{fig:yolo}. 

Multiple other versions and modifications are possible. A smaller and faster version is called \textit{Fast YOLO}. It has a similar architecture but uses only 9 convolutional layers. Another possibility to improve \textit{YOLO} is to replace the custom architecture with a more common feature extractor from a classification network. \textit{YOLO} build on top of a \textit{VGG-16} achieves better precision at the cost of half of the frames per second.

\begin{figure}
    \centering
    \includegraphics[width=\textwidth]{img/yoylo}
    \caption{\textit{YOLO} architecture for evaluating \textit{PASCAL VOC}. It uses 7 by 7 grid with 2 bboxes per cell. Detecting 20 categories, the output's shape is 7\x7\x30. From \cite[fig. 3]{bib:yolo}.}
    \label{fig:yolo} 
\end{figure}

\subsubsection{Training}
Although the network can be trained end-to-end, it is common for CNN to pre-train on \textit{ImageNet} dataset. This is also the case for \textit{YOLO}. First, convolutional layers are pre-trained on the \textit{ImageNet} dataset, then the detection layers are added, and the whole network is trained for detection. The model is optimized using the sum-squared error between predictions and ground-truths. The loss function is a sum of three parts, classification loss, localization loss, and bbox confidence loss. 

\paragraph{Classification loss}
\begin{align*}
\mathbf{L_{cls}} = \sum_{i=0}^{S^2}\mathbbm{1}_i^{\text{obj}} \sum_{c\in \text{class}} (p_i(c) - \hat{p}_i(c))^2
\end{align*}

\paragraph{Localization loss}
\begin{align*}
\mathbf{L_{loc}} &= \lambda_{\text{coord}} \sum_{i=0}^{S^2} \sum_{j=0}^B \mathbbm{1}_{ij}^{\text{obj}} \left[(x_i - \hat{x}_i)^2 + (y_i - \hat{y}_i)^2\right] \\
 &+  \lambda_{\text{coord}} \sum_{i=0}^{S^2} \sum_{j=0}^B \mathbbm{1}_{ij}^{\text{obj}} \left[(\sqrt{w_i} - \sqrt{\hat{w}_i})^2 + (\sqrt{h_i} - \sqrt{\hat{h}_i)^2}\right]
\end{align*}

\paragraph{Confidence loss}
\begin{align*}
\mathbf{L_{cnf}} &= \sum_{i=0}^{S^2} \sum_{j=0}^B \mathbbm{1}_{ij}^{\text{obj}} (C_i - \hat{C}_i)^2 
+ \lambda_{\text{noobj}} \sum_{i=0}^{S^2} \sum_{j=0}^B \mathbbm{1}_{ij}^{\text{noobj}} (C_i - \hat{C}_i)^2
\end{align*}

\noindent where $\mathbbm{1}^{\text{obj}}_i$ denotes if object appears in cell $i$ and $\mathbbm{1}^{\text{obj}}_{ij}$ denotes that the $j$th bounding box predictor in cell $i$ is responsible for that prediction.

The gradient of cells that do contain the object can be overpowered with the cells that do not.  Therefore, the loss from negative confidence predictions is decreased by $\lambda_{\text{noobj}} = 0.5$. And to emphasize the bbox predictions, localization loss is increased using $\lambda_{\text{coord}} = 5$.In the localization loss, we can see that the center coordinates are handled differently to width and height. The square root of width and height is used to equalize the impact of the absolute value of error in small and large boxes.


\subsubsection{Properties}
A primary virtue of \textit{YOLO} is its speed for real-time applications, and its simple architecture allows for easy training and end-to-end optimization. \textit{YOLO}'s detection layer is provided with context from the whole image which leads to less false detections than in region proposal methods. 

On the other hand, a significant problem with \textit{YOLO}'s grid-based detection system, is a limitation to one class per cell. This limitation results in the inability to detect multiple objects in close proximity, such as people in the crowd. 

\textit{YOLO} also suffers from multiple problems with precise localization. It learns to detect arbitrary shapes, which can be hard to generalize to objects in new and unusual aspect ratios. Also, it predicts the bboxes on the heavily down-sampled image which leads to further imprecision. 


\subsection{SSD: Single Shot MultiBox Detector (2016)}
\label{sec:ssd}
\textit{SSD} is another real-time detector, aiming to outperform \textit{Faster R-CNN} by using a single network with one time evaluation. \citeauthor{bib:ssd} \cite{bib:ssd} presented this model just months after \textit{YOLO}. Even thought these networks are build on similar principles, there are multiple key differences. \textit{SSD} manages to outperform \textit{Faster R-CNN} and \textit{YOLO} both in speed and precision.

\subsubsection{Detection}
One of the main features of \textit{SSD} is that the detector network is fully convolutional and does not utilize any fully connected layers. Predictions are therefore generated for every position of a convolutional window. \textit{SSD} adopts similar concept to \textit{Faster R-CNN's} anchor boxes, this time called \textit{default} or \textit{prior} boxes. For each position of the feature map, multiple prior boxes with different aspect ratios are proposed. By default, 6 boxes are used. Contrary to \textit{YOLO}, both bbox and classification predictions are made for each position and each prior box.

Bboxes are predicted relative to prior box location which are themselves relative to feature map location. There is no bbox or region confidence value. Instead, \textit{SSD} uses an additional background class in classification predictions. Considering \textit{B} prior boxes and \textit{C} classes, \textit{SSD} generates $m\times n\times B\times (C+5)$ parameters on feature map of m\x n size.

\textit{SSD} detects objects on multiple feature maps at different scales, to accommodate detection of different sized objects. Moreover, this allows detectors on each level to focus on predicting a smaller range of bbox sizes. For the illustration of this process see \cref{fig:ssddet}.

\begin{figure}
    \centering
    \includegraphics[width=\textwidth]{img/ssddet}
    \caption{SSD detection. (a) Input image with ground-truth boxes. (b) and (c) Predictions based on prior boxes on multiple scales of feature maps.}
    \label{fig:ssddet}
\end{figure}

\subsubsection{Architecture}
The \textit{SSD} architecture can be described as a set of three modules. A base network, extra convolutional layers and detection layers.

\begin{itemize}
    \item \textbf{Base network}'s task is to take the input image and produce a feature map. To this end, a feature extractor build from classification network is an ideal candidate. \textit{SSDs} base is build from \textit{VGG-16} network with some modifications. First of all, all fully connected layers are removed and replaced with another pair of convolution layers. \textit{Pool5} layer is also changed from 2\x2/2 to 3\x3/1 pooling. Technically, \textit{SSD} does not force any constraints on the base network, and any CNN can replace the \textit{VGG}.
    \item \textbf{Extra layers} serve the purpose of providing more feature maps on decreasing scale to the detector. Smaller feature maps aggregate more information to a smaller area and allow for detection of larger objects with small convolutional window. On the other hand, information about small objects can be lost. Therefore the use of gradually decreasing series of feature maps. \textit{Extra layers} are implemented as a sequence of convolution layers connected to the end of the \textit{base}.
    \item \textbf{Detection layers} are the final layers of the network. There is a pair of classification and localization convolutions for each feature map. Considering \textit{SSD300}, where 300 stands for the width and height of input image. Detection is performed on 6 feature maps of sizes [38, 19, 10, 5, 3, 1], using [4, 6, 6, 6, 4, 4] prior boxes respectively, producing 8732 predictions per class. First, two of those feature maps are pulled from the \textit{VGG} network, and the \textit{extra layers} provide the rest. All detection layers are implemented using 3\x3 convolutions with an appropriate number of filters, as seen on \cref{fig:VGGSSD}.
\end{itemize}

\begin{figure}
    \VGGSSD
    \caption{SSD architecture based on modified \textit{VGG-16} network. \textit{VGGs} three fully connected layers are replaced with two new convolutional layers and \textit{Pool5} layer is changed from 2\x2/2 to 3\x3/1 pooling. Detection on the first and last two feature maps uses 4 prior boxes, while the rest uses 6 boxes. See more details on \textit{VGG} in \cref{sec:VGG}.}
    \label{fig:VGGSSD}
\end{figure}


\subsubsection{Training}
\textit{SSD} pre-trains the \textit{base network} on \textit{ImageNet} dataset, and after that removes the classification layers and replaces them with \textit{extra} and \textit{detection} layers. The model is than trained for detection end-to-end. \textit{SSD} utilizes \textit{smooth L1} loss for localization and \textit{cross-entropy} loss for classification. The final loss is the sum of those two components.

Before the training, we need to figure out which prior boxes match the ground-truth annotations. For each ground-truth box, two criteria are used: a prior box with highest IoU is selected, and then, the ground-truth box is also matched to all prior-boxes with IoU higher than a threshold (0.5). Let the $x_{ij}^p = {0,1}$ be an indicator that $i$th prior box matches $j$th ground-truth box with class $p$.

To keep the balance between positive and negative samples, \textit{hard negative mining} algorithm is employed. Only top negative samples, with the highest confidence score, are chosen. The goal is to keep the ratio of positives and negatives below 1:3.

\paragraph{Localization loss} expresses the error between predicted boxes $(l)$ and ground-truths $(g)$. Predictions are generated in respect to corresponding prior boxes. Therefore, after matching the boxes, a ground-truths also need to be represented in respect to prior box $(d)$, with center $(c_x,c_y)$ and width $(w)$ and height $(h)$.

\begin{align*}
\mathbf{L_{\text{loc}}}(x,l,g) = \sum_{i\in Pos}^N \sum_{m\in(c_x, c_y, w, h)} &x_{ij}^k\text{smooth}_{L1}(l_i^m-\hat{g}_j^m) \\
\hat{g}_j^{c_x} = (g_j^{c_x} - d_i^{c_x}) / d_i^{w} \qquad& \hat{g}_j^{c_y} = (g_j^{c_y} - d_i^{c_y}) / d_i^{h} \\
\hat{g}_j^{w} = log(\frac{g_j^{w}}{d_i^w}) \qquad& \hat{g}_j^{h} = log(\frac{g_j^{h}}{d_i^h})
\end{align*}

\paragraph{Confidence loss} or classification loss, is the softmax loss over class confidences $(c)$.
\begin{align*}
\mathbf{L_{\text{cls}}}(x,c) = -\sum_{i\in Pos}^N x_{ij}^p log(\hat{c}_i^p) - \sum_{i \in Neg} log(\hat{c}_i^0) \quad\text{where} \quad\hat{c}_i^p = \frac{exp(c_i^p)}{\sum_p exp(c_i^p)}
\end{align*}

\noindent The final loss is then a weighted sum of both losses. The weight parameter $\alpha$ is set to 1. The loss is also divided by the number of matched prior boxes to keep it independent of the number of objects.

\begin{align*}
\mathbf{L}(x,c,l,g) = \frac{1}{N}(\mathbf{L_{\text{cls}}}(x,c) + \alpha\mathbf{L_{\text{loc}}}(x,l,g))
\end{align*}

\subsection{YOLOv2 (2017)}
\citeauthor{bib:yolo9000} \cite{bib:yolo9000}


\subsection{200fps}
\todo{ https://arxiv.org/pdf/1805.06361.pdf}

\section{video smthing}
\subsection{Optimizing Video Object Detection via a Scale-Time Lattice}

\chapter{Contributions}

\section{Improving the SSD}

\section{SSD-TC: SSD with temporal convolution}




\chapter{Experiments}
In this chapter, we provide more details on techniques used to gather the presented results. We also present more detailed and experimental results that lead us to the conclusions in previous chapter.

The original implementation provided with the research paper implements SSD in the \textit{Caffe} \footnote{\url{http://caffe.berkeleyvision.org/}} deep learning framework. We decided against performing our experiments in Caffe and decided to implement our version in newer \textit{PyTorch} \footnote{\url{https://pytorch.org/}} framework. 


\section{Xception}
In \cref{sec:fixxception} we introduced a hypothesis about Xception\textit{A}-SSD, and based on this hypothesis we presented the improved, Xception\textit{H} model. However, as suggested by the name, it was not our first modification, and we needed some trial-and-error testing to achieve this result.

We will describe every iteration of the Xception-SSD model we trained and the reasoning behind the individual modifications. For clarity, we will refer to the Xception\textit{X} models in this section only by their version letter. The performance of mentioned models on \textit{Surveillance dataset} is plotted on \cref{fig:xception_perf} and the feature map sizes inside the models are shown in \cref{tab:xmods}.


\begin{figure}
    \centering
    \includegraphics[width=0.95\textwidth]{img/fps_map_x}
    \caption[Performance of multiple Xception modification on Surveillance dataset]{Performance of multiple Xception modification on Surveillance dataset. Circle radii demonstrate relative difference of network parameter counts.} 
    \label{fig:xception_perf}
\end{figure}

\begin{table}[]
    \centering
    \begin{tabular}{c|c|c|c|c|c}
            &   A   &   B (C, D)   &   E (F, G)   &   H   &   J   \\
        \hline
        B1  &   [74\x128]   &   [74\x128]   &   [74\x128]   &   [74\x128]   &   [74\x128]   \\
        B2  &   \textbf{[37\x256]}   &  [37\x256]   &   [37\x256]   &   [37\x256]   &   [37\x256]   \\
        B3  &   [19\x728]   &   [19\x256]   &   [37\x256]   &   [37\x256]   &   [37\x256] \\
        \hline
        B4-6  &   [19\x728]   &   [19\x256]   &   [37\x256]   &   [37\x256]   &   [37\x256] \\
        B7  &   [19\x728]   &   \textbf{[19\x256]}   &   \textbf{[37\x256]}   &   \textbf{[37\x256]}   &   \textbf{[37\x256]} \\
        B8-10  &   [19\x728]   &   [19\x728]   &   [19\x728]   &   [19\x512]   &   [19\x512]\\
        B11 &   \textbf{[19\x728]}   &   \textbf{[19\x728]}   &   \textbf{[19\x728]}   &   \textbf{[19\x512]}   &   \textbf{[19\x512]}\\
        \hline
        B12 &   [10\x1024]  &   [10\x1024]  &   [10\x1024]  &   [10\x728]  &   [10\x512]\\
        S1  &   [10\x1536]  &   [10\x1536]  &   [10\x1536]  &   [10\x1024]  &   [10\x512]\\
        S2  &   \textbf{[10\x2048]}  &   \textbf{[10\x2048]}  &  \textbf{[10\x2048]}   &  \textbf{[10\x1024]} &  \textbf{[10\x512]}\\
    \end{tabular}
    \caption{Feature map sizes on the output of layers of the Xception networks. Bx stand for Xception blocks, S1 and S2 stand for separable convolution layers that follow the block structure (see \cref{fig:xception}). The first number represents spatial dimensions of a square feature, expecting [300\x300] input, and the second one represents the number of channels. The highlighted feature maps are used for the detections. The versions C, D and F, G share the feature maps sizes with their parent versions, if the given layers are present, but do not match the highlighted feature extraction.}
    \label{tab:xmods}
\end{table}

\subsubsection{Versions B, C, D}
We will take a look of this trio at once because version \textit{C} adds modifications to version \textit{B}, and version \textit{D} further modifies version \textit{C}. Notably, we trained these models in parallel and therefore had no results from concurrent models to inform on the design.

Starting with the assumption that the problem of the version \textit{A} is the location of the first feature extraction for detection, we moved the extraction from \textit{block 2} to \textit{block 7}. We also reduce the number of output filters on \textit{blocks 2 to 7} from 728 to 256. As a result the first detection is performed on a [19\x19\x256] feature map as opposed to original [37\x37\x256] map, but deeper in the network. 

Due to a reduction of feature map size, we considered this modification a half-measure in our plan to move the first feature extraction deeper into do network. However it proved to be a successful step in the right direction. The network has significantly gained in both speed and precision values. 

Both \textit{C} and \textit{D} were designed to observe the impact of removal of parts of the network and in no way modify parametrization of layers in the network. Version \textit{C} omits \textit{blocks 5, 6 and 7} and extracts the [19\x19] feature map after \textit{block 10} instead of \textit{block 11}. Version \textit{D}, was designed to test the need for 6 detection layers, mainly the layers for detecting large objects. It is based on \textit{C}, and removes the feature layers produced by \textit{extra layers}.

In version \textit{C}, we did not observe satisfying performance boost to justify received loss of precision. The version \textit{D}'s performance boost was more significant, however, it was coupled with major precision penalty.

\subsubsection{Versions E, F, G}
Similarly to the previous trio, these version are also based on each other, with \textit{E} being the main architectural change and versions \textit{F} and \textit{G} more of an independent experiments. 

Version \textit{E} is the one, where we finally implemented our original intention of moving first extracted feature map of [37\x37] size to a deeper layer. The only difference from \textit{B}, implementation-wise, is the movement of max-pooling with stride 2 from \textit{block 2} to \textit{block 8}. 

We can see a noticeable drop-off in performance compared to \textit{B}. Although the number of parameters in the network is the same, the change from [19\x19] to [37\x37] requires about 4 times more computation per convolutional layer. 


\subsubsection{Version H}

\subsubsection{Version J}


A - ap on person, car low


\section{Measurements}
To make our work comparable to other similar studies and future works, we present methods used for both precision and performance measurements.

\subsection{Precision}
Rather than implementing our own copy of the precision evaluation, all our precision measurements were taken using an external tool. Using the external tool, independent on our implementation, allows us to easily compare our results with other works with little to no modifications. We used the implementation by \citet{bib:metricsgit}, that mirrors the evaluation process of \textit{PASCAL VOC Challenge}. The default parameters were used, meaning that the interpolation of AP was calculated using all data points and the IoU threshold was set to 0.5. 

We present all the measurements taken on \textit{Surveillance dataset} while performing multiple experiments described in this thesis.


\begin{figure}
    \centering
    \includegraphics[width=\textwidth]{img/ap}
    \caption[Average precision of all tested networks on Surveillance classes]{Average precision of all tested networks on Surveillance classes. \textit{COCO} indicates that the network was trained on COCO dataset, otherwise only Surveillance data were used for training.} 
    \label{fig:ap}
\end{figure}

\subsection{Inference speed}
The absolute values of the inference speed measurement have no information value without the knowledge of the environment in which they have been taken. In this section, we provide the details of both software and hardware environments used for measurements.

The testing was done by timing the total time to process 10 000 images in batches of 16, and then simply calculating the \textit{fps} value. We do not consider scaling and cropping the images to be the part of the network and therefore we had no need to include this process in the measurement, the set of [300\x300] pixel images was pre-loaded into memory. On the other hand, the non-maximum suppression is critical part of the algorithm and is included. 

\paragraph{Hardware} All our testing was done on the following hardware:
\begin{itemize}
    \item AMD EPYC 7401P CPU @ 2GHz \x 24
    \item NVIDIA GeForce GTX 1080 Ti
    \item 128GB DDR4 RAM
\end{itemize}

\section{Training SSDTC}
\chapter*{Conclusion}
\addcontentsline{toc}{chapter}{Conclusion}

Our work can be summarized in five segments. We started by reviewing related work and other relevant image processing deep learning models. We continued by weighting the options for improving SSD and decided to replace VGG feature extractor by more modern network. Then we tool a look at relationship between detected classes and detector performance. In the forth part, we returned to improving the SSD, this time we picked one base network and instead of using the classification network as is, we make a series of adjustments aimed at improving the performance. We dedicated the final part of this work to designing and implementing a version of SSD detector with the use of temporal information from video continuity.

\subsubsection*{Model review}
We began by briefly summarizing a development of classification networks by presenting a selection of models while describing their architecture and historical significance. With the bases of image processing covered, we moved on to a region-based object detection and compiled a brief overview of R-CNN family of networks.

Since the focus of this thesis is to optimize the SSD detector for video surveillance, we presented an in-depth examination of this detector. However, it would not be right to present SSD without mentioning its contemporary counterpart, YOLO. Because both SSD and YOLO are one-stage detectors and our experience shows that one-stage approach can be difficult to comprehend, we compiled a detailed examination of such detector.

In order to gain wider knowledge about video detection and inspiration for our work, we also examined a pair of video detectors exploiting the temporal information in the video.

\subsubsection*{SSD's Base Network}
We know, from the research paper introducing SSD, that the precision of the model can be improved by implementing a sophisticated data augmentation algorithm. Since augmentation algorithms slow down the training process, we decided to advance with a fast and straightforward training and focus on the relative comparison of the network performances. To obtain a benchmark baseline, we began our work by re-implementing SSD in PyTorch framework and trained it on COCO dataset.

Our first steps towards optimizing SSD lead to a search for a replacement for the outdated VGG classifier. We tried out ResNet networks, Xception and NASNet-Mobile. The testing revealed that all versions of SSD based on ResNet are capable of outperforming VGG16-SSD, but Xception and NASNet only outperform VGG in terms of speed. This result suggested ResNet as a clear winner. However, we were not satisfied with all the results and further pursued the possibilities for improvement, mainly for the Xception-SSD. 

\subsubsection*{Classes}
Before we continued with testing of architecture modifications, we decided to take a side-step toward the training data. As stated before, our baseline is SSD trained on COCO dataset. However, we are not really interested in every class provided by COCO. As a matter of fact, we are only interested in seven of those eighty classes.

Instead of just training with only those selected classes, we took this opportunity to test a couple of hypothesis. 

formulated the hypothesis, according to which the precision of the detector can be negatively impacted by the removal of the unwanted classes from the training process. 

We performed two tests concerning limiting the number of detected classes, one experiment to test our hypothesis and therefore precision, and second test to observe the relationship between the number of detected classes and speed of the network.

The performance test was inconclusive and did not favor one dataset over the other. However, the inference speed test clearly shows the benefits of a lower number of classes. With ResNet50-SSD, we managed to speed up the network from 50 fps on 80 classes to 123 fps on 7 classes. Based on the results of those tests, we decided to continue with all further experiments with this limited dataset.

\subsubsection*{Architecture}
With training dataset taken care of, we returned to the Xception-SSD and its disappointing precision. As it usually goes with neural networks, we took an experimental approach and designed multiple versions of Xception to test. After a few iterations, we arrived at the Xception version H, that met our expectations. This version managed to perform on par with ResNet50-SSD, reaching over 49\% mAP on surveillance data and trailing by a few fps. 

\subsubsection*{TSSD}
TSSD



\section*{Future Work}






%%% Bibliography
\include{bibliography}

\appendix
\chapter{Implementation}
\label{app:impl}

environment
PyTorch 0.4.1
Python 3.5
CUDA Toolkit 9.2
cuDNN 7
NVIDIAs nvidia/cuda:9.2-cudnn7-devel docker container
opencv 3.4.3.18
Ubuntu 16.04



%%% Figures used in the thesis (consider if this is needed)
\listoffigures
% list of figures je dobrej do knizek co maj 500+ stranek, tady to je uplna zbytecnost.

%%% Tables used in the thesis (consider if this is needed)
%%% In mathematical theses, it could be better to move the list of tables to the beginning of the thesis.
%\listoftables

%%% Abbreviations used in the thesis, if any, including their explanation
%%% In mathematical theses, it could be better to move the list of abbreviations to the beginning of the thesis.
% \chapwithtoc{List of Abbreviations}
%%% Attachments to the bachelor thesis, if any. Each attachment must be
%%% referred to at least once from the text of the thesis. Attachments
%%% are numbered.
%%%
%%% The printed version should preferably contain attachments, which can be
%%% read (additional tables and charts, supplementary text, examples of
%%% program output, etc.). The electronic version is more suited for attachments
%%% which will likely be used in an electronic form rather than read (program
%%% source code, data files, interactive charts, etc.). Electronic attachments
%%% should be uploaded to SIS and optionally also included in the thesis on a~CD/DVD.
%%% Allowed file formats are specified in provision of the rector no. 23/2016.
%\chapwithtoc{Attachments}

\openright
\end{document}
