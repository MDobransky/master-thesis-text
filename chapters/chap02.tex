\chapter{Classification networks}
\label{chapt:cnets}
In order to better understand the detection networks, we will shortly describe their predecessors, classification networks. Classification network is a convolutional neural network (CNN) \cite[ch.~9]{bib:dlbook} which given an image, returns a confidence score that the image corresponds to each of the classes. In \cref{chapt:models} we will see that the classification models are used as a backbone for the detectors.

\section{Models}
\subsection{VGG}
\label{sec:VGG}
\cite{bib:vgg}

\begin{figure}
    \label{fig:resnet_arch}
    \centering
    \vggArch
    \caption{Architecture of VGG network, version D. Taken from \cite[table 1]{bib:vgg}}
\end{figure}

\subsection{Inception}
\label{sec:inception}


\subsection{ResNet}
\label{sec:resnet}

All ResNet \cite{bib:resnet} architectures (ResNet-10, ResNet-18, 34, 50...) use the same architecture (left chart), which is build from a few input layers, four sets of building blocks and output layers. Each of four blocks can be composed of multiple residual building blocks. There are two types of building blocks, larger block with three convolutional layers called "bottleneck", or smaller "basic" block with two convolutional layers.  see \cref{fig:resnet_arch}

\begin{figure}
    \label{fig:resnet_arch}
    \resnetArch
    \caption{Architecture of ResNet network.
    Note that the give number of convolutional filters in building blocks depends on whitch block do they belong to. For more detailed specification see Table 1 in  \cite{bib:resnet}.}
\end{figure}






