\chapter*{Conclusion}
\addcontentsline{toc}{chapter}{Conclusion}

Our work can be summarized in five segments. We started by reviewing related work and other relevant image processing deep learning models, and continued by weighting the options for improving SSD, where we decided to replace VGG feature extractor by more modern network. Then we tool a look at relationship between detected classes and detector performance. In the forth part, we returned to improving SSD, this time we focused on one base network and instead of using the classification network as is, we make a series of adjustments aimed at improving the performance. We dedicated the final part of this work to designing and implementing a version of SSD detector with the use of temporal information from video continuity.

\subsubsection*{Model review}
We began by briefly summarizing a development of classification networks by presenting a selection of models while describing their architecture and historical significance. After we covered the bases of image processing, we moved on to a region-based object detection and compiled a brief overview of R-CNN family of networks.

Since the focus of this thesis is to optimize the SSD detector for video surveillance, we presented an in-depth examination of this detector. However, it would not be right to present SSD without mentioning its contemporary counterpart, YOLO. Because both SSD and YOLO are one-stage detectors and our experience shows that one-stage approach can be difficult to comprehend, we compiled a detailed examination of this approach.

In order to gain wider knowledge about video detection and inspiration for our work, we also examined a pair of video detectors exploiting the temporal information in the video.

\subsubsection*{Feature Extraction Network}
We know, from the research paper introducing SSD, that the precision of the model can be improved by implementing a sophisticated data augmentation algorithm. Since augmentation algorithms slow down the training process, we decided to advance with a fast and simple augmentation and rather focus on the relative comparison of model performances. To obtain a benchmark baseline, we began our work by re-implementing SSD in PyTorch framework and trained it on COCO dataset.

Our first steps towards optimizing SSD lead to a search for a replacement for the underlying outdated VGG classifier. We tried out ResNet networks, Xception and NASNet-Mobile. The testing revealed that all versions of SSD based on ResNet are capable of outperforming standard SSD, but Xception and NASNet only outperform VGG in terms of speed. Although this result suggested ResNet as a clear winner, achieving 32.7\% mAP and 50fps over original 29.7\% mAP and 45fps, we were not satisfied with all the results and further pursued the possibilities for improvement, mainly for the Xception-SSD. 

\subsubsection*{Classes}
Before we continued with testing of architecture modifications, we decided to take a side-step toward the training data. As stated before, our baseline SSD is trained on COCO dataset. However, we are not really interested in every class provided by COCO. As a matter of fact, we are only interested in seven of those eighty classes.

Before we moved to this smaller dataset, we took this opportunity and performed two tests concerning the impact of limiting the number of detected classes. The first experiment was based on our hypothesis, according to which the removal of the unwanted classes from the training process can have a negative effect on the precision of the detector. The second test was to observe the relationship between the number of detected classes and speed of the network.

The results of precision test on were inconclusive for our test setup and did not favor one dataset over the other. However, the inference speed test clearly shows the benefits of a lower number of classes. With ResNet50-SSD, we managed to speed up the network from 50 fps on 80 classes to 123 fps on 7 classes. Based on the results of those tests, we decided to continue the further testing with this limited dataset.

\subsubsection*{Xception-SSD}
Although ResNet was the best performer in our first tests, we decided to try and improve the Xception-based SSD, since Xception looked promising in classification benchmarks and our SSD implementation showed disappointing precision. We already ruled out the option of keeping unmodified both SSD and Xception, therefore we decided to adjust the feature extraction part of the network to better fit the needs of SSD.

We took an experimental approach and designed multiple versions of Xception to test. After a few iterations, we arrived at the Xception version H. This version managed to reach 49.8\% mAP and perform on par with ResNet50-SSD's 48.7\% mAP while being about 15\% slower. 

Although this experiment did not result in the model that would manage to substantially outperform ResNet, we learned a lot about the relationship between classification models and detection ones. The fact is, that not every state-of-the-art classification network is fit to serve as a feature extractor for an object detector like SSD. Mainly if said detector uses multiple feature maps at different scales. Some networks, like NASNet-Mobile, may be outright unfit, and others needs adjustment to fit the needs of the detector.

\subsubsection*{TSSD}
TSSD



\section*{Future Work}




