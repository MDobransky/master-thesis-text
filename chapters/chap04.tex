\chapter{Contributions}

what made vgg ideal for ssds concept of detection on different scales was the gradual decrease in size of the feature map. We tried to mimic this behaviour by finding a suitably sized feature maps in other networks, which can complicate the implementation, especially in block architectures.


\section{Reimplement SSd on other base}
\subsection{ResNet-SSD}
\begin{figure}
    \label{fig:resnetSSD}
    \resnetSSD
    \caption{SSD detector build on ResNet base. Depth of classification convolutions depends on number of classes (C). For detailed description of \textit{Layers} see \cref{fig:resnet_arch}.}
\end{figure}

\subsection{Xception-SSD}


\begin{figure}
    \label{fig:xceptionSSD}
    \xceptionSSD
    \caption{SSD detector build on Xception base. Depth of classification convolutions depends on number of classes (C). Note that ReLU activation and batch normalisation functions are omitted from this chart. For detailed description of blocks see \cref{fig:xception}}
\end{figure}

\subsection{NasNet-SSD}

\begin{figure}
    \label{fig:nasnetSSD}
    \nasnetSSD
    \caption{SSD detector build on NASNet base. Depth of classification convolutions depends on number of classes (C). For detailed description of cells see \cref{sec:nasnet}}
\end{figure}


\section{Train only specific classes}
downside - categorizing animals as humans because it is most similar class.
\begin{table}
    \begin{tabular}{c|c|c}
         & coco & small \\
        Resnet50 & & 48.7\\
        Resnet101 & 32.8 & 45.74\\
        Xception & 26.76 & 37.83\\
        NASNet & & \\
    \end{tabular}
    \caption{mean average precision, conf thr 0.2, IoU 0.5,  (400 epochs), 300x300 scaled random crops}
    \label{tab:map}
\end{table}


\begin{table}
    \begin{tabular}{c|c|c}
     & 81cls & 8cls  \\
    Resnet34 & 31.1M & 24.4M\\
    Resnet50 & 45.5M & 28.7M\\
    Resnet101 & 64.5M & 47.7M \\
    Xception & 40.6M & 25.7M \\
    NASNet & 17.3M & 7.6M \\
    \end{tabular}
    \caption{total number of parameters}
    \label{tab:parameters}
\end{table}



\section{something, use video continuity???}