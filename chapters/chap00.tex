\chapter*{Introduction}
\addcontentsline{toc}{chapter}{Introduction}

In recent years, the video surveillance has become the most widely used security measure. Surveillance cameras can be found in many environments and used for multiple purposes such as traffic monitoring and marketing, but the primary reason is the security. As the more and more public space is covered by the surveillance, it has become more difficult to monitor all those video streams in real-time. A development in artificial intelligence and availability of computing power has created an obvious solution for this problem. Instead the people watching the streams, we can process the video data with the help of artificial intelligence and then output the desired notifications and statistics. 

Simple approaches like motion detection provide information about a whole video frame. These are relatively simple tasks and nowadays can be found implemented directly into the camera itself. However we need information about every object in the frame to be able to define exclusion zones, create heat maps, track the objects and observe their speed and direction. The last breakthrough in this area came with the application of artificial neural networks.

Given the set of pictures with annotations, the neural network can be trained to search for similar objects and categorize them properly. Current cutting edge detectors use convolutional neural network architectures and produce either set of bounding boxes or a pixel by pixel categorization of the desired objects. These detections can be then used and processed by other applications to produce desired statistics or notifications. 

The goal of this thesis is to review current techniques used to detect objects in video frames, implement and train a neural network designed for this purpose and measure its results in terms of precision and speed. We will also perform a series of experiments designed to boost the performance of our network with regards to our criteria. 

% \todo{ccv pipeline, datasets - augmentation problem}

% \paragraph{Related work}
% bla

% \paragraph{Approach}
% bla

% \paragraph{Thesis structure}
% bla
