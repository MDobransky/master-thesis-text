\chapter*{Introduction}
\addcontentsline{toc}{chapter}{Introduction}

In recent years, security cameras have become the most widely used surveillance measure. Other than for security, surveillance cameras are finding their use for multiple purposes such as traffic monitoring and marketing. Thanks to the wide range of applications, surveillance covers most of the public space in the cities. This amount of video streams has become very laborious to monitor. The obvious solution to this problem has presented itself with development in artificial intelligence and the broader availability of computing power. Instead of constant monitoring by people, artificial intelligence can process the video data and then output the desired notifications and statistics.

The technology in the area of automatization in video surveillance has been rapidly changing. The most elementary approaches like motion detection are based on the principle of frame difference. Frame difference methods either compare successive frames with each other or use a background subtraction. It is a fast and simple method for detection of moving objects and is nowadays commonly embedded directly in the security cameras. However, frame difference is prone to false detections as it detects every movement independently on the object type. 

Object detectors with classification are historically based on feature descriptors. The first such algorithm with competitive results was a face detector by \citeauthor{bib:viola}. A few years later, another breakthrough in detection came from \citeauthor{bib:hog} using the histogram of oriented gradients for human detection. The last major leap in the detection, the deep learning, has become the dominant approach in the current decade. In the ImageNet Large Scale Visual Recognition Challenge deep learning approaches have been consistently winning since 2012 and surpassed human performance in 2015.

Current cutting edge deep learning detectors use convolutional neural network architectures. Such networks produce either set of bounding boxes or a pixel by pixel categorization of the desired objects. These detection data can be then used and processed by other applications to produce desired statistics or notifications.

\subsubsection{Goals}
The goal of this thesis is to review state-of-the-art deep learning models used for object detection and implement a selected model. We aim to improve this model for purposes of surveillance by formulating hypotheses based on reviewed techniques and test them experimentally. We have the ambition to design and implement a real-time video detector with the use of temporal information.

Our focus is limited to the detector component of a more extensive detection pipeline. We are not concerned with the pre- and post-processing operations. Although many interesting tasks are based on object detection, e.g., tracking and re-identification,  they are not in the scope of this thesis.

\subsubsection{Thesis Structure}
\Cref{chap:nns} of this thesis begins by defining the metrics needed for evaluation of the performance of neural network detectors. We continue the definitions by presenting a notation used throughout this thesis. We present an overview of convolutional neural networks used for image classification and region based object detectors. 

\Cref{chap:related} is dedicated to provide an in-depth review of real-time detectors, namely SSD and YOLO, and a review of two methods for video detection with temporal information, Tube-CNN and Temporal SSD. 

We present our contributions in \cref{chap:contrib}. We propose a set of improvements of SSD aimed at real-time detection for video surveillance and test them experimentally. Our contributions include a comparison of SSD implemented on a multitude of base networks (ResNet, Xception, NASNet). Then, we tested the relationship between detector performance and a number of detected classes and created the Surveillance dataset for further testing. We also present an improved version of Xception, modified to suit the needs of SSD detector. Our final contribution is an extension of SSD detector by the addition of three-dimensional convolutions in the temporal dimension. We called our detector a Single Shot Detector with Temporal Convolution (SSDTC).

\Cref{chap:exp} provides details on the methodologies used in this work. It also presents supplementary results acquired during the experiments.