\chapter*{Introduction}
\addcontentsline{toc}{chapter}{Introduction}

In recent years, video surveillance has become the most widely used security measure. Surveillance cameras can be found in many environments and used for multiple purposes such as traffic monitoring and marketing, but the primary reason is security. As the surveillance covers the more and more public space, it has become more challenging to monitor all those video streams in real-time. A development in artificial intelligence and the availability of computing power has created an obvious solution for this problem. Instead of constant monitoring of video streams by people, we can process the video data with the help of artificial intelligence and then output the desired notifications and statistics. 

The technology in the area of automatization in video surveillance has been rapidly changing. The most straightforward approaches like motion detection are based on the principle of frame difference. Frame difference methods either compare successive frames with each other or use a background subtraction. It is a fast and simple method for detection of moving objects and is nowadays commonly embedded directly in the security cameras. However, frame difference is prone to false detections as it detects every movement independently on the object type. 

Object detectors with classification are historically based on feature descriptors. The first such algorithm with competitive results was a face detector by \citeauthor{bib:viola}. A few years later, another breakthrough in detection came from \citeauthor{bib:hog} using the histogram of oriented gradients for human detection. The last major leap in the detection, the deep learning, has become the dominant approach in the current decade. In \textit{ImageNet Large Scale Visual Recognition Challenge} deep learning approaches has been consistently winning since 2012 and eventually surpassed human performance in 2015.

\subsubsection{Goals}
Current cutting edge deep learning detectors use convolutional neural network architectures. Such networks produce either set of bounding boxes or a pixel by pixel categorization of the desired objects. These detection data can be then used and processed by other applications to produce desired statistics or notifications.

The goal of this thesis is to review the current deep learning models used for object detection and implement a selected model. We aim to improve this model for purposes of surveillance by formulating hypotheses based on reviewed techniques and test them experimentally. We have the ambition to design and implement a real-time video detector with the use of temporal information.

Our focus is limited to the detector component of a more extensive detection pipeline, and we are not concerned with the pre- and post-processing operations.  Although many interesting tasks are based on object detection, e.g., tracking and re-identification,  they are not in the scope of this thesis.

\subsubsection{Thesis structure}
