\chapter*{Introduction}
\addcontentsline{toc}{chapter}{Introduction}

In recent years, security cameras have become widely used for indoor and outdoor surveillance. Covering more and more public space in cities, the cameras start to serve various smart city purposes ranging from security to traffic monitoring and marketing. However, with the increasing quantity of utilized cameras and recorded streams, manual video monitoring and analysis becomes too laborious. Hence, developments in artificial intelligence and the broader availability of computing power are necessary to benefit from such rich data sources. Instead of monitoring by people, the goal is to train an effective and efficient artificial intelligence model to process the video data automatically and then output the desired notifications and statistics.

The automatization in video surveillance has evolved rapidly in the last decades. Not so long ago, state-of-the-art detection relied on motion detection based on the principle of frame difference. Frame difference methods either compare successive frames with each other or use background subtraction to detect changes in the video. This is a fast and simple method for detecting moving objects and can be nowadays commonly found embedded directly in the security cameras. However, the frame difference approach is prone to false detections as it is sensitive to every movement regardless of the object type. Hence, it is no longer used as a standalone detector.

An object detector with classification capability can not only enhance the detection by providing the means for class-based filtering but also removes a significant deficiency of frame differencing methods, detection of stationary objects.  Modern detectors gain this capability by relying on feature vectors for a class description. The first such algorithm with competitive real-time results was a face detector by \citeauthor{bib:viola}. A few years later, another significant step in detection came from \citeauthor{bib:hog} using the histogram of oriented gradients for human detection. The last major breakthrough in object detection that has become the dominant approach of the current decade is the application of deep learning. In the ImageNet Large Scale Visual Recognition Challenge deep learning approaches have been winning consistently since 2012 and surpassed human performance in 2015.


\subsubsection{Goals}
The goal of this thesis is to review state-of-the-art deep learning models for object detection and implement a real-time detection model based on the SSD approach. The objective is to not only re-implement a selected model but to improve that model for purposes of surveillance while using the information gathered from the review. This thesis aims to test proposed improvements experimentally. On top of the optimization of the SSD, we aspire to design and implement a second model, a real-time video detector with enhanced precision by exploiting the temporal information provided by the video.

The focus of this thesis is limited to the detection component of a more extensive video detection pipeline. We do not concern ourselves with optimizing pre- and post-processing operations. Although many interesting tasks are based on object detection, e.g., tracking and re-identification, they are not in the scope of this thesis.

\subsubsection{Thesis Structure}
\Cref{chap:prelim} defines the metrics needed for the evaluation of classification and object detection models, and notation used throughout this thesis. 

In \cref{chap:nns}, we present a brief overview of the history of convolutional neural networks used for image classification and region-based object detectors. 

\Cref{chap:related} is dedicated to providing an in-depth review of real-time detectors, namely SSD and YOLO, and a review of two methods for video detection with the use of temporal information, i.e., Tube-CNN and Temporal SSD. 

We present our contributions in \cref{chap:contrib}. We propose a set of improvements of SSD aimed at real-time detection for video surveillance and test them experimentally. Our contributions include a comparison of SSD implemented on a multitude of base networks (ResNet, Xception, NASNet), a test of the relationship between detector performance and a number of detected classes, and an improved version of Xception modified to suit the needs of SSD detector. Our final contribution is an extension of SSD detector by the addition of three-dimensional convolutions using the temporal dimension, called a Single Shot Detector with Temporal Convolution (SSDTC).

\Cref{chap:exp} provides details on the methodologies used in this work. It also presents supplementary results acquired during the experiments.